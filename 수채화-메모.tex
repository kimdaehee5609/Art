%	-------------------------------------------------------------------------------
%
%	2020년 6월 30일 첫 작
%
%
%
%
%
%
%	-------------------------------------------------------------------------------

%\documentclass[10pt,xcolor=pdftex,dvipsnames,table]{beamer}
%\documentclass[10pt,blue,xcolor=pdftex,dvipsnames,table,handout]{beamer}
%\documentclass[14pt,blue,xcolor=pdftex,dvipsnames,table,handout]{beamer}
\documentclass[aspectratio=1610,17pt,xcolor=pdftex,dvipsnames,table,handout]{beamer}

		% Font Size
		%	default font size : 11 pt
		%	8,9,10,11,12,14,17,20
		%
		% 	put frame titles 
		% 		1) 	slideatop
		%		2) 	slide centered
		%
		%	navigation bar
		% 		1)	compress
		%		2)	uncompressed
		%
		%	Color
		%		1) blue
		%		2) red
		%		3) brown
		%		4) black and white	
		%
		%	Output
		%		1)  	[default]	
		%		2)	[handout]		for PDF handouts
		%		3) 	[trans]		for PDF transparency
		%		4)	[notes=hide/show/only]

		%	Text and Math Font
		% 		1)	[sans]
		% 		2)	[sefif]
		%		3) 	[mathsans]
		%		4)	[mathserif]


		%	---------------------------------------------------------	
		%	슬라이드 크기 설정 ( 128mm X 96mm )
		%	---------------------------------------------------------	
%			\setbeamersize{text margin left=2mm}
%			\setbeamersize{text margin right=2mm}

	%	========================================================== 	Package
		\usepackage{kotex}						% 한글 사용
		\usepackage{amssymb,amsfonts,amsmath}	% 수학 수식 사용
		\usepackage{color}					%
		\usepackage{colortbl}					%


	%		========================================================= 	note 옵션인 
	%			\setbeameroption{show only notes}
		

	%		========================================================= 	Theme

		%	---------------------------------------------------------	
		%	전체 테마
		%	---------------------------------------------------------	
		%	테마 명명의 관례 : 도시 이름
%			\usetheme{default}			%
%			\usetheme{Madrid}    		%
%			\usetheme{CambridgeUS}    	% -red, no navigation bar
%			\usetheme{Antibes}			% -blueish, tree-like navigation bar

		%	----------------- table of contents in sidebar
			\usetheme{Berkeley}		% -blueish, table of contents in sidebar
									% 개인적으로 마음에 듬

%			\usetheme{Marburg}			% - sidebar on the right
%			\usetheme{Hannover}		% 왼쪽에 마크
%			\usetheme{Berlin}			% - navigation bar in the headline
%			\usetheme{Szeged}			% - navigation bar in the headline, horizontal lines
%			\usetheme{Malmoe}			% - section/subsection in the headline

%			\usetheme{Singapore}
%			\usetheme{Amsterdam}

		%	---------------------------------------------------------	
		%	색 테마
		%	---------------------------------------------------------	
%			\usecolortheme{albatross}	% 바탕 파란
%			\usecolortheme{crane}		% 바탕 흰색
%			\usecolortheme{beetle}		% 바탕 회색
%			\usecolortheme{dove}		% 전체적으로 흰색
%			\usecolortheme{fly}		% 전체적으로 회색
%			\usecolortheme{seagull}	% 휜색
%			\usecolortheme{wolverine}	& 제목이 노란색
%			\usecolortheme{beaver}

		%	---------------------------------------------------------	
		%	Inner Color Theme 			내부 색 테마 ( 블록의 색 )
		%	---------------------------------------------------------	

%			\usecolortheme{rose}		% 흰색
%			\usecolortheme{lily}		% 색 안 칠한다
%			\usecolortheme{orchid} 	% 진하게

		%	---------------------------------------------------------	
		%	Outter Color Theme 		외부 색 테마 ( 머리말, 고리말, 사이드바 )
		%	---------------------------------------------------------	

%			\usecolortheme{whale}		% 진하다
%			\usecolortheme{dolphin}	% 중간
%			\usecolortheme{seahorse}	% 연하다

		%	---------------------------------------------------------	
		%	Font Theme 				폰트 테마
		%	---------------------------------------------------------	
%			\usfonttheme{default}		
			\usefonttheme{serif}			
%			\usefonttheme{structurebold}			
%			\usefonttheme{structureitalicserif}			
%			\usefonttheme{structuresmallcapsserif}			



		%	---------------------------------------------------------	
		%	Inner Theme 				
		%	---------------------------------------------------------	

%			\useinnertheme{default}
			\useinnertheme{circles}		% 원문자			
%			\useinnertheme{rectangles}		% 사각문자			
%			\useinnertheme{rounded}			% 깨어짐
%			\useinnertheme{inmargin}			




		%	---------------------------------------------------------	
		%	이동 단추 삭제
		%	---------------------------------------------------------	
%			\setbeamertemplate{navigation symbols}{}

		%	---------------------------------------------------------	
		%	문서 정보 표시 꼬리말 적용
		%	---------------------------------------------------------	
%			\useoutertheme{infolines}


			
	%	---------------------------------------------------------- 	배경이미지 지정
%			\pgfdeclareimage[width=\paperwidth,height=\paperheight]{bgimage}{./fig/Chrysanthemum.jpg}
%			\setbeamertemplate{background canvas}{\pgfuseimage{bgimage}}

		%	---------------------------------------------------------	
		% 	본문 글꼴색 지정
		%	---------------------------------------------------------	
%			\setbeamercolor{normal text}{fg=purple}
%			\setbeamercolor{normal text}{fg=red!80}	% 숫자는 투명도 표시


		%	---------------------------------------------------------	
		%	itemize 모양 설정
		%	---------------------------------------------------------	
%			\setbeamertemplate{items}[ball]
%			\setbeamertemplate{items}[circle]
%			\setbeamertemplate{items}[rectangle]






		\setbeamercovered{dynamic}





		% --------------------------------- 	문서 기본 사항 설정
		\setcounter{secnumdepth}{5} 		% 문단 번호 깊이
		\setcounter{tocdepth}{5} 			% 문단 번호 깊이




% ------------------------------------------------------------------------------
% Begin document (Content goes below)
% ------------------------------------------------------------------------------
	\begin{document}
	

			\title{중앙도서관 수채화}

			\author{김대희}

			\date{2020년 06월 30일}


	%	==========================================================
	%		개정 이력
	%	----------------------------------------------------------
	%	2020.06.30 첫 작성
	%	----------------------------------------------------------
	%	
	%	----------------------------------------------------------


	%	==========================================================
	%
	%	----------------------------------------------------------
		\begin{frame}[plain]
		\titlepage
		\end{frame}



%		\begin{frame} [plain]{목차}
		\begin{frame} {목차}
		\tableofcontents
		\end{frame}
		

	%	========================================================== 	개요
	%		Frame
	%	----------------------------------------------------------
		\part{개요}
		\frame{\partpage}


		\begin{frame} [plain]{목차}
		\tableofcontents
		\end{frame}
		

		
				
		
	%	 ---------------------------------------------------------- 기본개요
	%	 Frame
	%	 ----------------------------------------------------------
		\section{기본 개요}
%		\frame [plain] {\sectionpage}
		

		\begin{frame} [t,plain]
			\begin{block} {기본 개요}

			\setlength{\leftmargini}{5em}			
			\begin{itemize}
				\item [강좌명]  
				\item [시간]  매주 화요일
				\item [장소]  소림사 소법당
				\item [연락처]  010 7595 9475
			\end{itemize}
			
			\end{block}
		\end{frame}
		



	%	 ---------------------------------------------------------- 주요 내용
	%	 Frame
	%	 ----------------------------------------------------------
		\section{주요내용}
%		\frame [plain] {\sectionpage}

	%	 ----------------------------------------------------------
		\begin{frame} [t,plain]
			\begin{block} {주요 내용}
			\begin{itemize}
				\item 수채화 도구 사용법 이해
				\item 색과 농도 기초 실습
				\item 정물 및 꽃이 핀 풍경 그리기 연습
				\item 자연 풍경 그리기 연습
			\end{itemize}
			\end{block}
		\end{frame}


	%	 ---------------------------------------------------------- 강사
	%	 Frame
	%	 ----------------------------------------------------------
		\section{강사}
%		\frame [plain] {\sectionpage}

	%	 ----------------------------------------------------------
		\begin{frame} [t,plain]
			\begin{block} {강사}
			\begin{itemize}
				\item 이종원
				\item 부산대 사법대학 미술교육과 졸업
				\item 미술1급 정교사
				\item 현, 고등학교 미술 교사
			\end{itemize}
			\end{block}
		\end{frame}


	%	========================================================== 함께 읽어볼 자료
		\part{ 함께 읽어볼 자료 }
		\frame{\partpage}
		
		\begin{frame} [plain]{목차}
		\tableofcontents%
		\end{frame}




	
	


	%	---------------------------------------------------------- 수채화로 그리는 꽃선물
	%		Frame
	%	----------------------------------------------------------
		\section{수채화로 그리는 꽃 선물 }
		\begin{frame} [t,plain]
		\frametitle{수채화로 그리는 꽃 선물 }
			\begin{block} {수채화로 그리는 꽃 선물 }
			\setlength{\leftmargini}{4em}			
			\begin{itemize}
				\item [제목] 수채화로 그리는 꽃 선물 \\: Watercolor guide book 기법서	
				\item [지은이] 박송연 지음	
				\item [출판사] EJONG(이종)	
				\item [중앙] 651-84 		
			\end{itemize}
			\end{block}						
								
		\end{frame}						

	%	---------------------------------------------------------- 수채화 원데이 클래스
	%		Frame
	%	----------------------------------------------------------
	
		\section{수채화 원데이 클래스}
		\begin{frame} [t,plain]
		\frametitle{수채화 원데이 클래스: 나의 첫 감성 수채화 노트} 	
			\begin{block} {수채화 원데이 클래스\\: 나의 첫 감성 수채화 노트} 	
			\setlength{\leftmargini}{4em}			
			\begin{itemize}
				\item [제목] 수채화 원데이 클래스\\: 나의 첫 감성 수채화 노트
				\item [지은이] 백초윤 지음	
				\item [출판사] 경향비피	
				\item [중앙] 652.52-23		
			\end{itemize}
			\end{block}						
								
		\end{frame}						


	%	---------------------------------------------------------- 공양주보살
	%		Frame
	%	----------------------------------------------------------

		\section{수채화 그리기 좋은 날 }	
		\begin{frame} [t,plain]
		\frametitle{수채화 그리기 좋은 날 : 13명의 작가와 함께하는 수채화 수업}	
			\begin{block} {수채화 그리기 좋은 날 \\: 13명의 작가와 함께하는 수채화 수업}	
			\setlength{\leftmargini}{4em}			
			\begin{itemize}
				\item [제목] 수채화 그리기 좋은 날 : 13명의 작가와 함께하는 수채화 수업
				\item [지은이] 이명선 외 지음	
				\item [출판사] 경향비피	
				\item [중앙] 652.52-27 		
			\end{itemize}
			\end{block}						
								
		\end{frame}						


	%	---------------------------------------------------------- 공양주보살
	%		Frame
	%	----------------------------------------------------------
 	
		\section{나도 수채화 잘 그리면 소원이 없겠네 }
		\begin{frame} [t,plain]
		\frametitle{나도 수채화 잘 그리면 소원이 없겠네 : 도구 사용법부터 꽃 그리기까지, 초보자를 위한 4주 클래스}
			\begin{block} {나도 수채화 잘 그리면 소원이 없겠네 \\: 도구 사용법부터 꽃 그리기까지, 초보자를 위한 4주 클래스}
			\setlength{\leftmargini}{4em}			
			\begin{itemize}
				\item [제목] 나도 수채화 잘 그리면 소원이 없겠네 : 도구 사용법부터 꽃 그리기까지, 초보자를 위한 4주 클래스
				\item [지은이] 차유정	
				\item [출판사] 한빛라이프	
				\item [중앙] 652.52-28		
			\end{itemize}
			\end{block}						
								
		\end{frame}						

	%	---------------------------------------------------------- 공양주보살
	%		Frame
	%	----------------------------------------------------------
	

		\section{ 나만의 수채화 교실	}
		\begin{frame} [t,plain]
		\frametitle{(매일 매일 행복을 느끼게 하는) 나만의 수채화 교실	}
			\begin{block} {(매일 매일 행복을 느끼게 하는) 나만의 수채화 교실	}
			\setlength{\leftmargini}{4em}			
			\begin{itemize}
				\item [제목] (매일 매일 행복을 느끼게 하는) 나만의 수채화 교실	
				\item [지은이] 윈저 지음	
				\item [출판사] 도도	
				\item [중앙] 652.52-29 		
			\end{itemize}
			\end{block}						
								
		\end{frame}						

	%	---------------------------------------------------------- 공양주보살
	%		Frame
	%	----------------------------------------------------------
		\section{오늘 본 수채화 풍경 }
		\begin{frame} [t,plain]
		\frametitle{오늘 본 수채화 풍경 : 7가지 기법으로 쉽게 그리는 30가지 풍경 수채화}
			\begin{block} {오늘 본 수채화 풍경 : 7가지 기법으로 쉽게 그리는 30가지 풍경 수채화}
			\setlength{\leftmargini}{4em}			
			\begin{itemize}
				\item [제목] 오늘 본 수채화 풍경 : 7가지 기법으로 쉽게 그리는 30가지 풍경 수채화
				\item [지은이] 김소라	
				\item [출판사] 책밥	
				\item [중앙] 652.52-30		
			\end{itemize}
			\end{block}						
								
		\end{frame}						

	%	---------------------------------------------------------- 공양주보살
	%		Frame
	%	----------------------------------------------------------
		

		\section{(다시 시작하는) 수채화 기초 클래스}
		\begin{frame} [t,plain]
		\frametitle{(다시 시작하는) 수채화 기초 클래스}
			\begin{block} {(다시 시작하는) 수채화 기초 클래스}
			\setlength{\leftmargini}{4em}			
			\begin{itemize}
				\item [제목] (다시 시작하는) 수채화 기초 클래스
				\item [지은이] 이수경 지음	
				\item [출판사] 이종	
				\item [중앙] 652.52-35 		
			\end{itemize}
			\end{block}						
								
		\end{frame}						

	%	---------------------------------------------------------- 공양주보살
	%		Frame
	%	----------------------------------------------------------
		\section{일러스트레이터 로사의 식물 수채화 가이드 }
		\begin{frame} [t,plain]
		\frametitle{물빛 보태니컬 가든 : 일러스트레이터 로사의 식물 수채화 가이드 }
			\begin{block} {물빛 보태니컬 가든 : 일러스트레이터 로사의 식물 수채화 가이드 }
			\setlength{\leftmargini}{4em}			
			\begin{itemize}
				\item [제목] 물빛 보태니컬 가든 : 일러스트레이터 로사의 식물 수채화 가이드 
				\item [지은이] 로사 지음	
				\item [출판사] 북핀	
				\item [중앙] 652.52-36		
			\end{itemize}
			\end{block}						
								
		\end{frame}						

	%	---------------------------------------------------------- 공양주보살
	%		Frame
	%	----------------------------------------------------------
		\section{	맛있는 수채화 일러스트레이션	}
		\begin{frame} [t,plain]
		\frametitle{	맛있는 수채화 일러스트레이션	}
			\begin{block} {	맛있는 수채화 일러스트레이션	}
			\setlength{\leftmargini}{4em}			
			\begin{itemize}
				\item [제목] 	맛있는 수채화 일러스트레이션	
				\item [지은이] 정원재 저	
				\item [출판사] 투데이북스	
				\item [중앙] 657.5-18		
			\end{itemize}
			\end{block}						
								
		\end{frame}						

	%	---------------------------------------------------------- 공양주보살
	%		Frame
	%	----------------------------------------------------------

		\section{	(꼭 배우고 싶은) 수채화 캘리그라피	 }
		\begin{frame} [t,plain]
		\frametitle{	(꼭 배우고 싶은) 수채화 캘리그라피	 }
			\begin{block} {	(꼭 배우고 싶은) 수채화 캘리그라피	 }
			\setlength{\leftmargini}{4em}			
			\begin{itemize}
				\item [제목] 	(꼭 배우고 싶은) 수채화 캘리그라피	 
				\item [지은이] 이명선 외 7인 지음	
				\item [출판사] 경향미디어	
				\item [중앙] 658.31-15		
			\end{itemize}
			\end{block}						
								
		\end{frame}						




	%	========================================================== 강의 준비물
		\part{강의 준비물}
		\frame{\partpage}
		
		\begin{frame} [plain]{목차}
		\tableofcontents%
		\end{frame}
		

% ----------------------------------------------------------------------------- 강의 준비물
%
% -----------------------------------------------------------------------------
	\section{강의 준비물 }
	\frame [plain] {\sectionpage}


	%	 ---------------------------------------------------------- 준비물
	%	 Frame
	%	 ----------------------------------------------------------
		\section{준비물}
%		\frame [plain] {\sectionpage}

	%	 ----------------------------------------------------------
		\begin{frame} [t,plain]
			\begin{block} {준비물}
			\begin{itemize}
				\item 물감
				\item 파레트
				\item 스케치북
				\item 붓
				\item 걸레
				\item 연필
				\item 지우개
			\end{itemize}
			\end{block}
		\end{frame}

	%	 ---------------------------------------------------------- 준비물
	%	 Frame
	%	 ----------------------------------------------------------
		\section{준비물}
%		\frame [plain] {\sectionpage}

	%	 ----------------------------------------------------------
		\begin{frame} [t,plain]
			\begin{block} {준비물}
			\begin{itemize}
				\item 물감
				\item 파레트
				\item 스케치북
				\item 붓
				\item 걸레
				\item 연필
				\item 지우개
			\end{itemize}
			\end{block}
		\end{frame}

% ----------------------------------------------------------------------------- 강의 준비물 : 붓
%
% -----------------------------------------------------------------------------
	\section{강의 준비물 : 붓}
	\frame [plain] {\sectionpage}


	%	========================================================== 수업
		\part{ 수업 }
		\frame{\partpage}
		
		\begin{frame} [plain]{목차}
		\tableofcontents%
		\end{frame}


	%	 ----------------------------------------------------------
		\begin{frame} [t,plain]
			\begin{block} {수업 : 2020년 7월 04일 토}

			\setlength{\leftmargini}{5em}			
			\begin{itemize}
				\item [출석] 
				\item [수업내용] 
				\item [반성문] 
			\end{itemize}
			\end{block}
		\end{frame}


	%	 ----------------------------------------------------------
		\begin{frame} [t,plain]
			\begin{block} {수업 : 2020년 7월 11일 토}

			\setlength{\leftmargini}{5em}			
			\begin{itemize}
				\item [출석] 
				\item [수업내용] 
				\item [반성문] 
			\end{itemize}
			\end{block}
		\end{frame}

	%	 ----------------------------------------------------------
		\begin{frame} [t,plain]
			\begin{block} {수업 : 2020년 7월 18일 토}

			\setlength{\leftmargini}{5em}			
			\begin{itemize}
				\item [출석] 
				\item [수업내용] 
				\item [반성문] 
			\end{itemize}
			\end{block}
		\end{frame}

	%	 ----------------------------------------------------------
		\begin{frame} [t,plain]
			\begin{block} {수업 : 2020년 7월 25일 토}

			\setlength{\leftmargini}{5em}			
			\begin{itemize}
				\item [출석] 
				\item [수업내용] 
				\item [반성문] 
			\end{itemize}
			\end{block}
		\end{frame}

	%	 ---------------------------------------------------------- 8월
		\begin{frame} [t,plain]
			\begin{block} {수업 : 2020년 8월 01일 토}

			\setlength{\leftmargini}{5em}			
			\begin{itemize}
				\item [출석] 
				\item [수업내용] 
				\item [반성문] 
			\end{itemize}
			\end{block}
		\end{frame}

	%	 ---------------------------------------------------------- 8월
		\begin{frame} [t,plain]
			\begin{block} {수업 : 2020년 8월 08일 토}

			\setlength{\leftmargini}{5em}			
			\begin{itemize}
				\item [출석] 
				\item [수업내용] 
				\item [반성문] 
			\end{itemize}
			\end{block}
		\end{frame}


	%	 ---------------------------------------------------------- 8월
		\begin{frame} [t,plain]
			\begin{block} {수업 : 2020년 8월 15일 토}

			\setlength{\leftmargini}{5em}			
			\begin{itemize}
				\item [출석] 
				\item [수업내용] 
				\item [반성문] 
			\end{itemize}
			\end{block}
		\end{frame}

	%	 ---------------------------------------------------------- 8월
		\begin{frame} [t,plain]
			\begin{block} {수업 : 2020년 8월 22일 토}

			\setlength{\leftmargini}{5em}			
			\begin{itemize}
				\item [출석] 
				\item [수업내용] 
				\item [반성문] 
			\end{itemize}
			\end{block}
		\end{frame}
		
	%	 ---------------------------------------------------------- 8월
		\begin{frame} [t,plain]
			\begin{block} {수업 : 2020년 8월 29일 토}

			\setlength{\leftmargini}{5em}			
			\begin{itemize}
				\item [출석] 
				\item [수업내용] 
				\item [반성문] 
			\end{itemize}
			\end{block}
		\end{frame}

% ------------------------------------------------------------------------------ ------------------------------------------------------------------------------ ------------------------------------------------------------------------------
% End document
% ------------------------------------------------------------------------------ ------------------------------------------------------------------------------ ------------------------------------------------------------------------------
\end{document}


	%	----------------------------------------------------------
	%		Frame
	%	----------------------------------------------------------
		\begin{frame} [c]
%		\begin{frame} [b]
%		\begin{frame} [t]
		\frametitle{감리 보고서}
		\end{frame}						

