%	------------------------------------------------------------------------------
%
%		작성  : 2020년 7월 17일 첫 작업
%
%

%	\documentclass[25pt, a1paper]{tikzposter}
%	\documentclass[25pt, a0paper, landscape]{tikzposter}
%	\documentclass[20pt, a1paper ]{tikzposter}
	\documentclass[	20pt, 
							a0paper, 
%							portrait, %
							landscape, %
							margin=0mm, %
							innermargin=4mm,  		%
							blockverticalspace=4mm, %
							colspace=5mm, 
							subcolspace=0mm
							]{tikzposter}



%	\documentclass[25pt, a1paper]{tikzposter}
%	\documentclass[25pt, a1paper]{tikzposter}
%	\documentclass[25pt, a1paper]{tikzposter}

% 	12pt  14pt 17pt  20pt  25pt
%
%	a0 a1 a2
%
%	landscape  portrait
%

	%% Tikzposter is highly customizable: please see
	%% https://bitbucket.org/surmann/tikzposter/downloads/styleguide.pdf

	%	========================================================== 	Package
		\usepackage{kotex}						% 한글 사용


%% Available themes: see also
%% https://bitbucket.org/surmann/tikzposter/downloads/themes.pdf
%	\usetheme{Default}
%	\usetheme{Rays}
%	\usetheme{Basic}
	\usetheme{Simple}
%	\usetheme{Envelope}
%	\usetheme{Wave}
%	\usetheme{Board}
%	\usetheme{Autumn}
%	\usetheme{Desert}

%% Further changes to the title etc is possible
%	\usetitlestyle{Default}			%
%	\usetitlestyle{Basic}				%
%	\usetitlestyle{Empty}				%
%	\usetitlestyle{Filled}				%
%	\usetitlestyle{Envelope}			%
%	\usetitlestyle{Wave}				%
%	\usetitlestyle{verticalShading}	%


%	\usebackgroundstyle{Default}
%	\usebackgroundstyle{Rays}
%	\usebackgroundstyle{VerticalGradation}
%	\usebackgroundstyle{BottomVerticalGradation}
%	\usebackgroundstyle{Empty}

%	\useblockstyle{Default}
%	\useblockstyle{Basic}
%	\useblockstyle{Minimal}		% 이것은 간단함
%	\useblockstyle{Envelope}		% 
%	\useblockstyle{Corner}		% 사각형
%	\useblockstyle{Slide}			%	띠모양  
	\useblockstyle{TornOut}		% 손그림모양


	\usenotestyle{Default}
%	\usenotestyle{Corner}
%	\usenotestyle{VerticalShading}
%	\usenotestyle{Sticky}

%	\usepackage{fontspec}
%	\setmainfont{FreeSerif}
%	\setsansfont{FreeSans}

%	------------------------------------------------------------------------------ 제목

	\title{ 수채화 }

	\author{ 	작성 : 2020년7월 17일 \\
			수정 : 2022년 8월 12일 금요일 }


	%% Optional title graphic
	%\titlegraphic{\includegraphics[width=7cm]{IMG_1934}}
	%% Uncomment to switch off tikzposter footer
	% \tikzposterlatexaffectionproofoff

\begin{document}


	\maketitle[
					width=841mm,
					linewidth = 2mm,
					innersep=4mm,
%					titletotopverticalspace=0mm, %
%					titletoblockverticalspace=0mm, %
					titletextscale =4, 
				]

		%		a0  841 - 1189
		%		a1  594 - 841
		%		a2  420 - 594



	\begin{columns}

		\column{0.2}

%	------------------------------------------------------------------------------ 준비물
			\block{■  준비물 }
			{
%					\setlength{\leftmargini}{7em}
%					\setlength{\labelsep} {1em}
				\begin{LARGE}
					\begin{itemize}
					\item [1.] 용지
					\item [2.] 물감
					\item [3.] 붓
					\item [4.] 연필 과 지우재
					\item [5.] 팔레트
					\item [6.] 물통
					\item [7.] 수건
					\item [8.] 마스킹액

					\end{itemize}
				\end{LARGE}
			}


%	------------------------------------------------------------------------------
		\block{■	용지			}
		{
%			\setlength{\leftmargini}{4em}			
%			\setlength{\labelsep}{1em} % horizontal space from bullet to text (as needed)
%			\setlength{\partopsep}{0pt}
%			\setlength{\parskip}{0pt}
%			\setlength{\parsep}{0pt}
%			\setlength{\topsep}{0pt}
%			\setlength{\itemsep}{0pt}
%			\setlength{\itemindent}{4em} % summary indentation: par 1.25cm + bullet 0.63 + magic number picked empirically (0.25cm)
%			\setlength{\leftmargin}{0pt}			

			\begin{LARGE}
			\begin{itemize}
			\item 황목 : 압축이 가장 덜 된 종이
			\item 중목 : 중간 정도 입축된 종이
			\item 세목 : 가장 강하게 압축된 종이
			\item 중성지
			\item 버킹포드 : 100\% 셀로로스로 만들어진 수채화지
			\item 켄솔 몽발
			\item 머메이드
			\item 물을 많이 사용하ㅣ므로 300g 이상을 사용
			\item 초보자의 경우 25\% 정도의 코튼 함유량을 가진 종이를선택
			\end{itemize}
			\end{LARGE}

		}T


%	------------------------------------------------------------------------------
		\block{■	물감			}
		{
%			\setlength{\leftmargini}{4em}			
%			\setlength{\labelsep}{1em} % horizontal space from bullet to text (as needed)
%			\setlength{\partopsep}{0pt}
%			\setlength{\parskip}{0pt}
%			\setlength{\parsep}{0pt}
%			\setlength{\topsep}{0pt}
%			\setlength{\itemsep}{0pt}
%			\setlength{\itemindent}{4em} % summary indentation: par 1.25cm + bullet 0.63 + magic number picked empirically (0.25cm)
%			\setlength{\leftmargin}{0pt}			

			\begin{LARGE}
			\begin{itemize}
			\item 신한 물감
			\item 18색 세트
			\end{itemize}
			\end{LARGE}

		}

%	------------------------------------------------------------------------------
		\block{■	붓			}
		{
%			\setlength{\leftmargini}{2em}			
%			\setlength{\labelsep}{1em} % horizontal space from bullet to text (as needed)
%			\setlength{\partopsep}{0pt}
%			\setlength{\parskip}{0pt}
%			\setlength{\parsep}{0pt}
%			\setlength{\topsep}{0pt}
%			\setlength{\itemsep}{0pt}
%			\setlength{\itemindent}{4em} % summary indentation: par 1.25cm + bullet 0.63 + magic number picked empirically (0.25cm)
%			\setlength{\leftmargin}{0pt}			

			\begin{LARGE}
			\begin{itemize}
			\item 바바라 붓 : 탄력이 좋아 다루기 쉬우며 세밀한 작업에 용이하다.
			\item 화홍 : 전반적으로 부드럽다. 털이 빨리 닳는다. 물조절이 잘 된다
			\item 루벤스 : 

			\end{itemize}
			\end{LARGE}
		}


	%	====== ====== ====== ====== ====== 
		\column{0.2}

%	------------------------------------------------------------------------------
		\block{■	연필 과 지우개			}
		{
			\setlength{\leftmargini}{2em}			
			\setlength{\labelsep}{1em} % horizontal space from bullet to text (as needed)
%			\setlength{\partopsep}{0pt}
%			\setlength{\parskip}{0pt}
%			\setlength{\parsep}{0pt}
%			\setlength{\topsep}{0pt}
%			\setlength{\itemsep}{0pt}
%			\setlength{\itemindent}{4em} % summary indentation: par 1.25cm + bullet 0.63 + magic number picked empirically (0.25cm)
%			\setlength{\leftmargin}{0pt}			

			\begin{LARGE}
			\begin{itemize}
			\item 연필은 스케치가 번지지 않도록 HB를 사용한다. 
			\item 샤프를 사용해도 좋다.
			\item 지우개는 말랑한 지우재를  사용
			\item 수채화 전용 연필 AQUARELLE
			\end{itemize}
			\end{LARGE}
		}


%	------------------------------------------------------------------------------
		\block{■	팔레트			}
		{
			\setlength{\leftmargini}{2em}			
			\setlength{\labelsep}{1em} % horizontal space from bullet to text (as needed)
%			\setlength{\partopsep}{0pt}
%			\setlength{\parskip}{0pt}
%			\setlength{\parsep}{0pt}
%			\setlength{\topsep}{0pt}
%			\setlength{\itemsep}{0pt}
%			\setlength{\itemindent}{4em} % summary indentation: par 1.25cm + bullet 0.63 + magic number picked empirically (0.25cm)
%			\setlength{\leftmargin}{0pt}			

			\begin{LARGE}
			\begin{itemize}
			\item 물감 짜는 순서
			\item 물감 짜는 요령
			\end{itemize}
			\end{LARGE}
		}


%	------------------------------------------------------------------------------
		\block{■	물통			}

%	------------------------------------------------------------------------------
		\block{■	수건			}


%	------------------------------------------------------------------------------
		\block{■	마스킹 액			}

%	------------------------------------------------------------------------------
		\block{■	마스킹 테이프		}


%	------------------------------------------------------------------------------
		\block{■	흡수 패드			}


	%	====== ====== ====== ====== ====== 
		\column{0.2}

		\block{■ 채색하기 기초 }


%	------------------------------------------------------------------------------
		\block{■ 물감 농도 조절하기 }

		\block{■ 색상표 만들기 }

		\block{■ 그라데이션 }

		\block{■ 물감 농도에 따른 번짐의 차이 }


		\block{■ 색 겹침 }

		\block{■ 번지기 }




	%	====== ====== ====== ====== ====== 
		\column{0.2}

%	------------------------------------------------------------------------------
		\block{■ 색상표 만들기 }
		{
			\setlength{\leftmargini}{4em}			
			\setlength{\labelsep}{1em} % horizontal space from bullet to text (as needed)
%			\setlength{\partopsep}{0pt}
%			\setlength{\parskip}{0pt}
%			\setlength{\parsep}{0pt}
%			\setlength{\topsep}{0pt}
%			\setlength{\itemsep}{0pt}
%			\setlength{\itemindent}{4em} % summary indentation: par 1.25cm + bullet 0.63 + magic number picked empirically (0.25cm)
%			\setlength{\leftmargin}{0pt}			

			\begin{LARGE}
			\begin{itemize}
			\item [청] 위창 오창석 1844-1927
			\item [청] 제백석
			\item [근대] 소세경
			\end{itemize}
			\end{LARGE}

		}

%	------------------------------------------------------------------------------
		\block{■ 발색표 만들기 }



	%	====== ====== ====== ====== ====== 
		\column{0.2}

		\block{■ 채색하기 기초 }


%	------------------------------------------------------------------------------ 수채화 배우기 전에 }
		\block{■ 수채화 배우기 전에 }
			{				
%			\setlength{\leftmargini}{9em}			
			\setlength{\labelsep}{1em} % horizontal space from bullet to text (as needed)

			\begin{LARGE}
			\begin{itemize}
			\item 아직은 잘 그리기 못하는 것을 인정하자
			\item 재료의 성질을 파악해 보자
			\item 잘 그린 그림에 대한 기준을 다시 세우자
			\item 물의 성질을 파악하지
			\item 여러가지 새깔을 다양하게 섞어보자
			\item 많이 망쳐보자
			\item 낙서를 많이 하자
			\item 쯜겁게 그리자
			\end{itemize}
			\end{LARGE}
		}


%	------------------------------------------------------------------------------ 자주 받는 질문
		\block{■ 자주 받는 질문 }
			{				
%			\setlength{\leftmargini}{9em}			
			\setlength{\labelsep}{1em} % horizontal space from bullet to text (as needed)

			\begin{LARGE}
			\begin{itemize}
			\item 
			\item 
			\item 
			\item 
			\item 
			\item 
			\end{itemize}
			\end{LARGE}
		}



	\end{columns}




\end{document}


		\begin{huge}
		\end{huge}

		\begin{LARGE}
		\end{LARGE}

		\begin{Large}
		\end{Large}

		\begin{large}
		\end{large}

