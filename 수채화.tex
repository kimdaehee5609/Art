%	-------------------------------------------------------------------------------
% 
%		2020년 6월 28일 첫작업
%
%
%
%
%
%
%
%	-------------------------------------------------------------------------------

%	\documentclass[12pt, a3paper, oneside]{book}
	\documentclass[12pt, a4paper, oneside]{book}
%	\documentclass[12pt, a4paper, landscape, oneside]{book}

		% --------------------------------- 페이지 스타일 지정
		\usepackage{geometry}
%		\geometry{landscape=true	}
		\geometry{top 			=10em}
		\geometry{bottom			=10em}
		\geometry{left			=8em}
		\geometry{right			=8em}
		\geometry{headheight		=4em} % 머리말 설치 높이
		\geometry{headsep		=2em} % 머리말의 본문과의 띠우기 크기
		\geometry{footskip		=4em} % 꼬리말의 본문과의 띠우기 크기
% 		\geometry{showframe}
	
%		paperwidth 	= left + width + right (1)
%		paperheight 	= top + height + bottom (2)
%		width 		= textwidth (+ marginparsep + marginparwidth) (3)
%		height 		= textheight (+ headheight + headsep + footskip) (4)



		%	===================================================================
		%	package
		%	===================================================================
%			\usepackage[hangul]{kotex}				% 한글 사용
			\usepackage{kotex}					% 한글 사용
			\usepackage[unicode]{hyperref}			% 한글 하이퍼링크 사용

		% ------------------------------ 수학 수식
			\usepackage{amssymb,amsfonts,amsmath}	% 수학 수식 사용
			\usepackage{mathtools}				% amsmath 확장판

			\usepackage{scrextend}				% 
		

		% ------------------------------ LIST
			\usepackage{enumerate}			%
			\usepackage{enumitem}			%
			\usepackage{tablists}				%	수학문제의 보기 등을 표현하는데 사용
										%	tabenum


		% ------------------------------ table 
			\usepackage{longtable}			%
			\usepackage{tabularx}			%
			\usepackage{tabu}				%




		% ------------------------------ 
			\usepackage{setspace}			%
			\usepackage{booktabs}		% table
			\usepackage{color}			%
			\usepackage{multirow}			%
			\usepackage{boxedminipage}	% 미니 페이지
			\usepackage[pdftex]{graphicx}	% 그림 사용
			\usepackage[final]{pdfpages}		% pdf 사용
			\usepackage{framed}			% pdf 사용

			
			\usepackage{fix-cm}	
			\usepackage[english]{babel}
	
		%	=======================================================================================
		% 	tikz package
		% 	
		% 	--------------------------------- 	
			\usepackage{tikz}%
			\usetikzlibrary{arrows,positioning,shapes}
			\usetikzlibrary{mindmap}			
			

		% --------------------------------- 	page
			\usepackage{afterpage}		% 다음페이지가 나온면 어떻게 하라는 명령 정의 패키지
%			\usepackage{fullpage}			% 잘못 사용하면 다 흐트러짐 주의해서 사용
%			\usepackage{pdflscape}		% 
			\usepackage{lscape}			%	 


			\usepackage{blindtext}
	
		% --------------------------------- font 사용
			\usepackage{pifont}				%
			\usepackage{textcomp}
			\usepackage{gensymb}
			\usepackage{marvosym}



		% Package --------------------------------- 

			\usepackage{tablists}				%


		% Package --------------------------------- 
			\usepackage[framemethod=TikZ]{mdframed}				% md framed package
			\usepackage{smartdiagram}								% smart diagram package



		% Package ---------------------------------    연습문제 

			\usepackage{exsheets}				%

			\SetupExSheets{solution/print=true}
			\SetupExSheets{question/type=exam}
			\SetupExSheets[points]{name=point,name-plural=points}


		% --------------------------------- 페이지 스타일 지정

		\usepackage[Sonny]		{fncychap}

			\makeatletter
			\ChNameVar	{\Large\bf}
			\ChNumVar	{\Huge\bf}
			\ChTitleVar		{\Large\bf}
			\ChRuleWidth	{0.5pt}
			\makeatother

%		\usepackage[Lenny]		{fncychap}
%		\usepackage[Glenn]		{fncychap}
%		\usepackage[Conny]		{fncychap}
%		\usepackage[Rejne]		{fncychap}
%		\usepackage[Bjarne]	{fncychap}
%		\usepackage[Bjornstrup]{fncychap}

		\usepackage{fancyhdr}
		\pagestyle{fancy}
		\fancyhead{} % clear all fields
		\fancyhead[LO]{\footnotesize \leftmark}
		\fancyhead[RE]{\footnotesize \leftmark}
		\fancyfoot{} % clear all fields
		\fancyfoot[LE,RO]{\large \thepage}
		%\fancyfoot[CO,CE]{\empty}
		\renewcommand{\headrulewidth}{1.0pt}
		\renewcommand{\footrulewidth}{0.4pt}
	
	
	
		%	--------------------------------------------------------------------------------------- 
		% 	tritlesec package
		% 	
		% 	
		% 	------------------------------------------------------------------ section 스타일 지정
	
			\usepackage{titlesec}
		
		% 	----------------------------------------------------------------- section 글자 모양 설정
			\titleformat*{\section}					{\large\bfseries}
			\titleformat*{\subsection}				{\normalsize\bfseries}
			\titleformat*{\subsubsection}			{\normalsize\bfseries}
			\titleformat*{\paragraph}				{\normalsize\bfseries}
			\titleformat*{\subparagraph}				{\normalsize\bfseries}
	
		% 	----------------------------------------------------------------- section 번호 설정
			\renewcommand{\thepart}				{\arabic{part}.}
			\renewcommand{\thesection}				{\arabic{section}.}
			\renewcommand{\thesubsection}			{\thesection\arabic{subsection}.}
			\renewcommand{\thesubsubsection}		{\thesubsection\arabic{subsubsection}}
			\renewcommand\theparagraph 			{$\blacksquare$ \hspace{3pt}}

		% 	----------------------------------------------------------------- section 페이지 나누기 설정
			\let\stdsection\section
			\renewcommand\section{\newpage\stdsection}



		%	--------------------------------------------------------------------------------------- 
		% 	\titlespacing*{commandi} {left} {before-sep} {after-sep} [right-sep]		
		% 	left
		%	before-sep		:  수직 전 간격
		% 	after-sep	 	:  수직으로 후 간격
		%	right-sep

			\titlespacing*{\section} 			{0pt}{1.0em}{1.0em}
			\titlespacing*{\subsection}	  		{0ex}{1.0em}{1.0em}
			\titlespacing*{\subsubsection}		{0ex}{1.0em}{1.0em}
			\titlespacing*{\paragraph}			{0em}{1.5em}{1.0em}
			\titlespacing*{\subparagraph}		{4em}{1.0em}{1.0em}
	
		%	\titlespacing*{\section} 			{0pt}{0.0\baselineskip}{0.0\baselineskip}
		%	\titlespacing*{\subsection}	  		{0ex}{0.0\baselineskip}{0.0\baselineskip}
		%	\titlespacing*{\subsubsection}		{6ex}{0.0\baselineskip}{0.0\baselineskip}
		%	\titlespacing*{\paragraph}			{6pt}{0.0\baselineskip}{0.0\baselineskip}
	

		% --------------------------------- recommend		섹션별 페이지 상단 여백
		\newcommand{\SectionMargin}				{\newpage  \null \vskip 2cm}
		\newcommand{\SubSectionMargin}			{\newpage  \null \vskip 2cm}
		\newcommand{\SubSubSectionMargin}		{\newpage  \null \vskip 2cm}


		%	--------------------------------------------------------------------------------------- 
		% 	toc 설정  - table of contents
		% 	
		% 	
		% 	----------------------------------------------------------------  문서 기본 사항 설정
			\setcounter{secnumdepth}{4} 		% 문단 번호 깊이
			\setcounter{tocdepth}{2} 			% 문단 번호 깊이 - 목차 출력시 출력 범위

			\setlength{\parindent}{0cm} 		% 문서 들여 쓰기를 하지 않는다.


		%	--------------------------------------------------------------------------------------- 
		% 	mini toc 설정
		% 	
		% 	
		% 	--------------------------------------------------------- 장의 목차  minitoc package
			\usepackage{minitoc}

			\setcounter{minitocdepth}{1}    	%  Show until subsubsections in minitoc
%			\setlength{\mtcindent}{12pt} 	% default 24pt
			\setlength{\mtcindent}{24pt} 	% default 24pt

		% 	--------------------------------------------------------- part toc
		%	\setcounter{parttocdepth}{2} 	%  default
			\setcounter{parttocdepth}{0}
		%	\setlength{\ptcindent}{0em}		%  default  목차 내용 들여 쓰기
			\setlength{\ptcindent}{0em}         


		% 	--------------------------------------------------------- section toc

			\renewcommand{\ptcfont}{\normalsize\rm} 		%  default
			\renewcommand{\ptcCfont}{\normalsize\bf} 	%  default
			\renewcommand{\ptcSfont}{\normalsize\rm} 	%  default


		%	=======================================================================================
		% 	tocloft package
		% 	
		% 	------------------------------------------ 목차의 목차 번호와 목차 사이의 간격 조정
			\usepackage{tocloft}

		% 	------------------------------------------ 목차의 내어쓰기 즉 왼쪽 마진 설정
			\setlength{\cftsecindent}{2em}			%  section

		% 	------------------------------------------ 목차의 목차 번호와 목차 사이의 간격 조정
			\setlength{\cftsecnumwidth}{2em}		%  section





		%	=======================================================================================
		% 	flowchart  package
		% 	
		% 	------------------------------------------ 목차의 목차 번호와 목차 사이의 간격 조정
			\usepackage{flowchart}
			\usetikzlibrary{arrows}


		%	=======================================================================================
		% 		makeindex package
		% 	
		% 	------------------------------------------ 목차의 목차 번호와 목차 사이의 간격 조정
%			\usepackage{makeindex}
%			\usepackage{makeidy}


		%	=======================================================================================
		% 		각주와 미주
		% 	

		\usepackage{endnotes} %미주 사용


		%	=======================================================================================
		% 	줄 간격 설정
		% 	
		% 	
		% 	--------------------------------- 	줄간격 설정
			\doublespace
%			\onehalfspace
%			\singlespace
		
		

	% 	============================================================================== itemi Global setting

	
		%	-------------------------------------------------------------------------------
		%		Vertical spacing
		%	-------------------------------------------------------------------------------
			\setlist[itemize]{topsep=0.0em}			% 상단의 여유치
			\setlist[itemize]{partopsep=0.0em}			% 
			\setlist[itemize]{parsep=0.0em}			% 
%			\setlist[itemize]{itemsep=0.0em}			% 
			\setlist[itemize]{noitemsep}				% 
			
		%	-------------------------------------------------------------------------------
		%		Horizontal spacing
		%	-------------------------------------------------------------------------------
			\setlist[itemize]{labelwidth=1em}			%  라벨의 표시 폭
			\setlist[itemize]{leftmargin=8em}			%  본문 까지의 왼쪽 여백  - 4em
			\setlist[itemize]{labelsep=3em} 			%  본문에서 라벨까지의 거리 -  3em
			\setlist[itemize]{rightmargin=0em}			% 오른쪽 여백  - 4em
			\setlist[itemize]{itemindent=0em} 			% 점 내민 거리 label sep 과 같은면 점위치 까지 내민다
			\setlist[itemize]{listparindent=3em}		% 본문 드려쓰기 간격
	
	
			\setlist[itemize]{ topsep=0.0em, 			%  상단의 여유치
						partopsep=0.0em, 		%  
						parsep=0.0em, 
						itemsep=0.0em, 
						labelwidth=1em, 
						leftmargin=2.5em,
						labelsep=2em,			%  본문에서 라벨 까지의 거리
						rightmargin=0em,		% 오른쪽 여백  - 4em
						itemindent=0em, 		% 점 내민 거리 label sep 과 같은면 점위치 까지 내민다
						listparindent=0em}		% 본문 드려쓰기 간격
	
%			\begin{itemize}
	
		%	-------------------------------------------------------------------------------
		%		Label
		%	-------------------------------------------------------------------------------
			\renewcommand{\labelitemi}{$\bullet$}
			\renewcommand{\labelitemii}{$\bullet$}
%			\renewcommand{\labelitemii}{$\cdot$}
			\renewcommand{\labelitemiii}{$\diamond$}
			\renewcommand{\labelitemiv}{$\ast$}		
	
%			\renewcommand{\labelitemi}{$\blacksquare$}   	% 사각형 - 찬것
%			\renewcommand\labelitemii{$\square$}		% 사각형 - 빈것	
			






% ------------------------------------------------------------------------------
% Begin document (Content goes below)
% ------------------------------------------------------------------------------
	\begin{document}
	
			\dominitoc
			\doparttoc			




			\title{ 중앙도서관 수채화 강좌 }
			\author{김대희}
			\date{2020년 6월}
			\maketitle


			\tableofcontents 		% 목차 출력
%			\listoffigures 			% 그림 목차 출력
			\cleardoublepage
			\listoftables 			% 표 목차 출력





		\mdfdefinestyle	{con_specification} {
						outerlinewidth		=1pt			,%
						innerlinewidth		=2pt			,%
						outerlinecolor		=blue!70!black	,%
						innerlinecolor		=white 			,%
						roundcorner			=4pt			,%
						skipabove			=1em 			,%
						skipbelow			=1em 			,%
						leftmargin			=0em			,%
						rightmargin			=0em			,%
						innertopmargin		=2em 			,%
						innerbottommargin 	=2em 			,%
						innerleftmargin		=1em 			,%
						innerrightmargin		=1em 			,%
						backgroundcolor		=gray!4			,%
						frametitlerule		=true 			,%
						frametitlerulecolor	=white			,%
						frametitlebackgroundcolor=black		,%
						frametitleaboveskip=1em 			,%
						frametitlebelowskip=1em 			,%
						frametitlefontcolor=white 			,%
						}



%	================================================================== Part			중앙도서관
	\addtocontents{toc}{\protect\newpage}
	\part{중앙도서관 수채화 강좌 }
	\noptcrule
	\parttoc				


%	================================================================== Part			개요
	\addtocontents{toc}{\protect\newpage}
	\chapter{개요}
	\noptcrule

	\newpage	
	\minitoc


% ----------------------------------------------------------------------------- 개요
%
% -----------------------------------------------------------------------------
	\section{ 수채화 강좌 개요}




\paragraph{ 프로그램명}
수채화

\paragraph{ 목표}

색과 농도의 이해, 수채화 채색법 기본교육으로 표현 방법을 익히고

다양한 풍경화를 완성할 수 있다

\paragraph{  주요 내용}


\begin{itemize}[					
		topsep=0.0em,			
		parsep=0.0em,			
		itemsep=0em,			
		leftmargin=	3	em,
		labelwidth=	1	em,			
		labelsep=		1	 em			
]					
	\item	수채화 도구 사용법 이해
	\item	색과 농도 기초 실습
	\item	정물 및 꽃이 핀 풍경 그리기 연습
	\item	자연 풍경 그리기 연습

\end{itemize}		


\paragraph{  강사}

\begin{itemize}[					
		topsep=0.0em,			
		parsep=0.0em,			
		itemsep=0em,			
		leftmargin=	3	em,
		labelwidth=	1	em,			
		labelsep=		1	 em			
]					
	\item	이종원
	\item	부산대학교 사범대학 미술교육학과 졸업
	\item	미술 1급 정교사
	\item	현, 고등학교 미술 교사

\end{itemize}		



\paragraph{  연락처 }

	중앙도서관 평행학습과 051)250-0322




% ----------------------------------------------------------------------------- 강의자
%
% -----------------------------------------------------------------------------
	\section{ 강의자 }  



\begin{itemize}[					
		topsep=0.0em,			
		parsep=0.0em,			
		itemsep=0em,			
		leftmargin=	3	em,
		labelwidth=	1	em,			
		labelsep=		1	 em			
]					
	\item	이종원
	\item	부산대학교 사범대학 미술교육학과 졸업
	\item	미술 1급 정교사
	\item	현, 고등학교 미술 교사

\end{itemize}				


% ----------------------------------------------------------------------------- 함께읽어볼자료
%
% -----------------------------------------------------------------------------
	\section{ 함께 읽어볼 자료 : 중앙도서관 } 


	% ----------------------------- table
	% 테이블에서 줄 간격 조정 

		\begin{table}[hbp]
		\caption{ 함께 읽어볼 자료 : 중앙 도서관 }
		\centering 

		\tabulinesep=4pt

		\begin{tabu} to \linewidth{ X[r,4] X[r,1] X[r,1] X[r,1]   }
			\toprule
			책 제목	&지은이	&출판사	&중앙 도서관\\
			\tabucline[1pt]{-}
수채화로 그리는 꽃 선물 : Watercolor guide book 기법서			&박송연 지음	&EJONG(이종)	&651-84 		\\
수채화 원데이 클래스: 나의 첫 감성 수채화 노트 				&백초윤 지음	&경향비피		&652.52-23	\\
수채화 그리기 좋은 날	: 13명의 작가와 함께하는 수채화 수업	&이명선 외 지음	&경향비피		&652.52-27 \\
나도 수채화 잘 그리면 소원이 없겠네 : 도구 사용법부터 꽃 그리기까지, 초보자를 위한 4주 클래스 	
														&차유정 지음	&한빛라이프	&652.52-28 \\
(매일 매일 행복을 느끼게 하는) 나만의 수채화 교실				&윈저 지음	&도도	&652.52-29 \\
오늘 본 수채화 풍경 : 7가지 기법으로 쉽게 그리는 30가지 풍경 수채화 
														&김소라 지음	&책밥	&652.52-30 \\
(다시 시작하는) 수채화 기초 클래스	&이수경 지음	&이종	&652.52-35 	\\
물빛 보태니컬 가든 : 일러스트레이터 로사의 식물 수채화 가이드	&로사 지음	&북핀	&652.52-36	\\
맛있는 수채화 일러스트레이션								&정원재 저	&투데이북스	&657.5-18	\\
(꼭 배우고 싶은) 수채화 캘리그라피							&이명선 외 7인 지음	&경향미디어	&658.31-15\\
			\bottomrule
		\end{tabu}

%		\label{table}
		\end{table}


% ----------------------------------------------------------------------------- 함께읽어볼자료
%
% -----------------------------------------------------------------------------
	\section{ 함께 읽어볼 자료 : 수정분관 } 


	% ----------------------------- table
	% 테이블에서 줄 간격 조정 

		\begin{table}[hbp]
		\caption{ 함께 읽어볼 자료 : 수정분관}
		\centering 

		\tabulinesep=4pt

		\begin{tabu} to \linewidth{ X[r,4] X[r,1] X[r,1] X[r,1]   }
			\toprule
			책 제목	&지은이	&출판사	&수정 분관  도서관\\
			\tabucline[1pt]{-}
수채화로 그리는 꽃 선물 : Watercolor guide book 기법서										&박송연 지음	&EJONG(이종)		&651-36 		\\
나도 수채화 잘 그리면 소원이 없겠네 : 도구 사용법부터 꽃 그리기까지, 초보자를 위한 4주 클래스 		&차유정 지음	&한빛라이프			&652.52-10 \\
오늘 본 수채화 풍경 : 7가지 기법으로 쉽게 그리는 30가지 풍경 수채화 								&김소라 지음	&책밥				&652.52-11 \\
(다시 시작하는) 수채화 기초 클래스															&이수경 지음	&이종				&652.52-15 	\\
맛있는 수채화 일러스트레이션																&정원재 저	&투데이북스			&657.5-13	\\
(꼭 배우고 싶은) 수채화 캘리그라피															&이명선 외 7인 지음	&경향미디어	&658.31-6\\
			\bottomrule
		\end{tabu}

%		\label{table}
		\end{table}





% ----------------------------------------------------------------------------- 강의준비물
%
% -----------------------------------------------------------------------------
	\section{ 강의 준비물 }




	% ----------------------------- table
	% 테이블에서 줄 간격 조정 

		\begin{table}[hbp]
		\caption{ 강의 준비물 }
		\centering 

		\tabulinesep=6pt

		\begin{tabu} to \linewidth{ X[r,1] X[r,4] X[c,3] X[c,2]  X[c,2] }
			\toprule
			번호		&명칭	&규격	&가격	 	&비고	 \\
			\tabucline[1pt]{-}
			1 	&수채화물감			&신한24색	&	 \\ \tabucline[1pt]{=}
			2 	&수채화용 파레트		&			&	 \\ \tabucline[1pt]{=}
			3 	&둥근붓 				&16호		&4	 \\ \tabucline[1pt]{=}
			4 	&둥근붓 				&80호		&4	 \\ \tabucline[1pt]{=}
			5 	&연필 				&HB			&4	 \\ \tabucline[1pt]{=}
			6 	&붓걸레 				&			&4	 \\ \tabucline[1pt]{=}
			7 	&미술용 지우개		&			&4	 \\ \tabucline[1pt]{=}
			8 	&수채화용 스케치북	&8절 300gr	&4	 \\
			\bottomrule
		\end{tabu}

%		\label{table}
		\end{table}



% ----------------------------------------------------------------------------- 강의계획
%
% -----------------------------------------------------------------------------
	\section{ 강의 계획 }



	% ----------------------------- table
	% 테이블에서 줄 간격 조정 

		\begin{table}[hbp]
		\caption{ 강의 계획 }
		\centering 

		\tabulinesep=3pt

		\begin{tabu} to \linewidth{ X[r,1] X[r,4] X[c,3]   X[c,2] }
			\toprule
			번호		&일자			&수업내용	 	&비고	 \\
			\tabucline[1pt]{-}
			1 		&7월 04일			&	&	 \\ \tabucline[1pt]{=}
			2 		&7월 11일			&	&	 \\ \tabucline[1pt]{=}
			3 		&7월 18일			&	&	 \\ \tabucline[1pt]{=}
			4 		&7월 25일			&휴강	&	 \\ \tabucline[1pt]{=}
			5 		&8월 01일			&	&	 \\ \tabucline[1pt]{=}
			6 		&8월 08일			&	&	 \\ \tabucline[1pt]{=}
			7 		&8월 15일			&	&	 \\ \tabucline[1pt]{=}
			8 		&8월 22일			&	&	 \\ \tabucline[1pt]{=}
			9 		&8월 29일			&	&	 \\ \tabucline[1pt]{=}
			10 		&9월 05일			&	&	 \\ \tabucline[1pt]{=}
			11 		&9월 12일			&	&	 \\ \tabucline[1pt]{=}
			12 		&9월 19일			&	&	 \\ \tabucline[1pt]{=}
			13 		&9월 26일			&	&	 \\ \tabucline[1pt]{=}
			14 		&10월 03일			&	&	 \\ \tabucline[1pt]{=}
			15 		&10월 10일			&	&	 \\ \tabucline[1pt]{=}
			16 		&10월 17일			&	&	 \\ \tabucline[1pt]{=}
			17 		&10월 24일			&	&	 \\ \tabucline[1pt]{=}
			18 		&10월 28일			&	&	 \\ \tabucline[1pt]{=}
			19 		&11월 07일			&	&	 \\ \tabucline[1pt]{=}
			20 		&11월 14일			&	&	 \\ \tabucline[1pt]{=}
			21 		&11월 21일			&	&	 \\ \tabucline[1pt]{=}
			22 		&11월 28일			&	&	 \\ \tabucline[1pt]{=}
			23 		&12월 05일			&	&	 \\ \tabucline[1pt]{=}

			\bottomrule
		\end{tabu}

%		\label{table}
		\end{table}



%	================================================================== Part			수업
	\addtocontents{toc}{\protect\newpage}
	\chapter{수업}
	\noptcrule

	\newpage	
	\minitoc




% ----------------------------------------------------------------------------- 외부링크
	\section{ 01 : 2020년 7월 4일 }

		\subsection{ 강의 내용}

		\subsection{ 반성문}













% ------------------------------------------------------------------------------
% End document
% ------------------------------------------------------------------------------
\end{document}


	\href{https://www.youtube.com/watch?v=SpqKCQZQBcc}{태양경배자세A}
	\href{https://www.youtube.com/watch?v=CL3czAIUDFY}{태양경배자세A}


https://docs.google.com/spreadsheets/d/1-wRuFU1OReWrtxkhaw9uh5mxouNYRP8YFgykMh2G_8c/edit#gid=0
+

https://seoyeongcokr-my.sharepoint.com/:f:/g/personal/02017_seoyoungeng_com/Ev8nnOI89D1LnYu90SGaVj0BTuckQ46vQe1HiVv-R4qeqQ?e=S3iAHi